\chapter{Initialisierung}

\begin{comment}
    Die Initialisierung schafft eine definierte Ausgangslage für das Projekt und stellt sicher, dass die
    Projektziele mit PkOrg übereinstimmen. Die Projektgrundlagen und der Projektauftrag sind
    erarbeitet. Es wird ein Variantenentscheid getroffen, welcher schlussendlich von den Lernenden
    realisiert und i.d.R eingeführt wird. Hinweis: Es ist sinnvoll, eine Risikoanalyse zu erstellen.
\end{comment}

\section{Studie Ist Zustand}
% Wie sieht der heutige Zustand aus? Ev. den Prozess abbilden.

\section{Vorgehensziele}
\begin{comment}
    bsp.
    Projekt beginn
    Projekt ende
    Zeitpensum
    Meilensteine erreichen im Projekt
    Projektmethode soll richtig angewendet werden
\end{comment}
\section{Systemziele}

\section{Anforderungen}

\section{Risikoanalyse}
\begin{comment}
    Welche Risiken ergeben sich, wenn das Projekt nicht realisiert wird? Was ist, wenn das Projekt scheitert? Welches sind die grössten Risiken bei diesem Projekt. (kann auch in den Anhang)
\end{comment}

\section{Varianten}
\begin{comment}
    Wichtig: Mögliche Varianten vorstellen und sich begründet für eine Variante entscheiden.
    Gibt es keine Varianten, so muss dies gemäss Kriterien-Katalog begründet werden.
\end{comment}

\section{Informationssicherheit und Datenschutz(ISDS)}
