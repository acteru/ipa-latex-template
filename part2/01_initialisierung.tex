\chapter{Initialisierung}

\begin{comment}
    Die Initialisierung schafft eine definierte Ausgangslage für das Projekt und stellt sicher, dass die
    Projektziele mit PkOrg übereinstimmen. Die Projektgrundlagen und der Projektauftrag sind
    erarbeitet. Es wird ein Variantenentscheid getroffen, welcher schlussendlich von den Lernenden
    realisiert und i.d.R eingeführt wird. Hinweis: Es ist sinnvoll, eine Risikoanalyse zu erstellen.
\end{comment}

%In diesem Abschnitt wird die Initialisierung behandelt und Abgeschlossen
\section{Studie Ist Zustand / Soll-Zustand}
\section{Vorgehensziele}
\begin{comment}
    bsp.
    Projekt beginn
    Projekt ende
    Zeitpensum
    Meilensteine erreichen im Project
    Projektmethode soll richtig angewendet werden
\end{comment}
\section{Systemziele}
\section{Anforderungen}
\section{Riskioanalyse}
\section{Varianten}
% Auswahlen der Varianten zum Projekt %
\section{Informationssicherheit und Datenschutz(ISDS)}
