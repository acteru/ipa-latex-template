\chapter{Einführung}

\begin{comment}
    Der sichere Übergang vom alten zum neuen Zustand wird gewährleistet. Der Betrieb wird ggf.
    aufgenommen und so lange durch das Projekt unterstützt, bis er stabil ist.
    Die Dokumentationen werden pünktlich auf Pkorg hochgeladen.
    Das Projekt wird abgeschlossen und die „Projektorganisation“ wird aufgelöst.
    Danach folgen die Präsentation und die anschliessende Bewertung durch die Experten und
    Fachvorgesetzten.
\end{comment}

% Hier können auch immer noch Tests durchgeführt werden
\section{Testkonzept und -infrastruktur überführen}
%Nach Projektabschluss werden für Korrekturen und Weiterentwicklungen Tests durchgeführt. Deshalb werden das Testkonzept und die Testinfrastr
\chapter{Quellenverzeichnis}
\begin{comment}
    Hinweis: An dieser Stelle muss ein Literatur- und Quellenverzeichnis eingefügt werden.
    (es kann auch mit der Fussnote ein Hinweis auf die Quelle gemacht werden, diese muss aber im Quellen VZ ersichtlich sein)

    Internet Quelle:
    Name des Autors (falls erkennbar), „Titel der Seite“, Webadresse, Datum des letzten Zugriffs

    Beispiel Internet-Quellenangabe:
    „Koala“, http://de.wikipedia.org/wiki/Koala, 22.03.2008

    Buch Quelle:
    Erklärung   Name des Autors: Titel. Verlag, Jahr, Seite, auf der der zitierte Text steht.

    Beispiel Buch-Quellenangabe:
    Ernst Walter Bauer: Humanbiologie. Cornelsen, 2006, S. 50.
\end{comment}


\chapter{Glossar}
% Alphabetisch sortiert

\begin{tabular}{l | l}
    \textbf{Begriff} & \textbf{Bedeutung} \\
    placeholder      & placeholder \\
\end{tabular}

\chapter{Unterschriften für Abnahme}
\begin{comment}
    Lernende und Fachvorgesetzte haben die Dokumentation vor der Abgabe zu unterzeichnen und somit die Richtigkeit zu bezeugen (gelbes Deckblatt). Der FV hat die IPA abgenommen, und bewertet diese anschliessend.

    Achtung:
    - Das gelbe Deckblatt muss sowohl vom Lernenden wie auch vom Fachvorgesetzten unterschrieben sein.
    - Die blauen Deckblätter (für die Exemplare der Experten) müssen nicht unterschrieben werden.
\end{comment}

\begin{tabularx}{0.9\textwidth}{| X | X | X |}
    \hline
    \textbf{Datum} & \textbf{Name/OE} & \textbf{Unterschrift} \\
    \hline
                   & & \\
                   & & \\
    \hline
                   & & \\
                   & & \\
    \hline
\end{tabularx}
