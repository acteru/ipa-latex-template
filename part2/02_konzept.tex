\chapter{Konzept}
\section{Detailkonzept}
\subsection{Lösungskonzept}
\begin{comment}
Hier muss schon relativ genau beschrieben werden wie das Netzwerk und die Hardware aussieht und wie die Lösung am Schluss abgeliefert werden soll.
\end{comment}
\subsection{Beschreibung der Variante}
\subsection{Anforderung an die Realisierung}
\subsection{Anforderungsabdeckung}
\section{Machbarkeitsanalyse}
\subsection{Wirtschaftliche Machbarkeit}
\subsection{Technische Machbarkeit}
\subsection{Ressourcen und Verfügbarkeit}
\subsection{Zeitliche Umsetzung}
\subsection{Empfehlung}
\section{Testkonzept}
\subsection{Testziel}
% Was soll dieser Test bringen und wie solle der grob Ablauf sein %
\subsection{Testobjekte}
% Objekt zb Apache %
\subsection{Testarten}
\subsection{Testabdeckung}
\subsection{Testrahmen}
\subsubsection{Testvorausetzungen}
\subsubsection{Fehlerklassen}
\subsubsection{Start- und Abbruchbedingungen}
\subsection{Testinfrastruktur}
\subsubsection{Testsystem}
\subsubsection{Testdaten}
\subsubsection{Testhilfsmittel}
\subsection{Testfallbeschreibung}
\subsection{Testplan}