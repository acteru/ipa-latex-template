\chapter{Realisierung}

\begin{comment}
    Das Produkt bzw. das IT-System wird realisiert und getestet. Die nötigen Vorarbeiten werden
    geleistet, um die Einführungsrisiken zu minimieren. Braucht es noch ein „Re-Testing“ oder werden
    mögliche kleine Fehler bei einem späteren Zeitpunkt noch korrigiert?
\end{comment}

\section{Realisierung Anwendung}
\subsection{Application}
\subsection{Application}
\subsection{Application}
\subsubsection{Application-subsection}
\subsubsection{Application-subsection}
\section{Tests durchführen}
\begin{comment}
    - Die Testdurchführung erfolgt erst, wenn die Vorbedingungen dazu erfüllt sind. Entsprechende muss vorher die Testinfrastruktur freigegeben werden.
    - Aktivitäten
    - Prüfen, ob die Testvorbedingunge erfüllt sind, um die Tests zu starten
    - Tests gemäss Testkonzept durchführen
    - Testergebnisse protokollieren und gemäss Kriterien im Testkonzept beurteilen
    - Gegebenfalls Fehler beheben und Tests wiederholen
    - Vorgehen zu offenen Punkten vereinbaren
\end{comment}
\section{Benutzeranleitung für Systemadministratoren}
