\chapter{Realisierung}
\begin{comment}
-Die Testinfrastruktur wird vor Beginn der Tests bereitgestellt. Sie umfasst alle Elemente, die für die Testdurchführung, die Sammlung und Bewertung der Testergebnisse notwendig sind.
-Grundidee
-Die Testinfrastruktur umfasst das Testsystem und die Testhilfsmittel (z.B Testmanagementsystem zur Sammlung und Bewertung der Ergebnisse) umd im weiteren Sinn die Testdaten.
-Aktivitäten
-Testinfrastruktur gemäss dem Testkonzept bereitstellen
-Qualität der Testinfrastruktur sicherstellen
-Testinfrastruktur für die Tests freigeben
\end{comment}
\section{Realisierung Anwendung}
\subsection{Application}
\subsection{Application}
\subsection{Application}
\subsubsection{Application-subsection}
\subsubsection{Application-subsection}
\section{Tests durchführen}
\begin{comment}
- Die Testdurchführung erfolgt erst, wenn die Vorbedingungen dazu erfüllt sind. Entsprechende muss vorher die Testinfrastruktur freigegeben werden.
- Aktivitäten
- Prüfen, ob die Testvorbedingunge erfüllt sind, um die Tests zu starten
- Tests gemäss Testkonzept durchführen
- Testergebnisse protokollieren und gemäss Kriterien im Testkonzept beurteilen
- Gegebenfalls Fehler beheben und Tests wiederholen
- Vorgehen zu offenen Punkten vereinbaren
\end{comment}
\section{Benutzeranleitung für Systemadministratoren}