\documentclass[a4paper,11pt,oneside]{report} % Lege Grundeigenschaften des Dokuments fest

% VARIABELN
\newcommand{\ipaAuthor}{Max Muster}
\newcommand{\ipaFachvorgesetzter}{Erika Mustermann}
\newcommand{\ipaFirma}{Company}


% PACKAGES
\usepackage[utf8]{inputenc} % Ermöglicht Umlaute usw...
\usepackage[ngerman]{babel} % Deutsch-Support
\usepackage{makeidx} % Benötigt man für das Inhaltsverzeichnis
\usepackage{hyperref} % Ermöglicht Hyperlinks
\usepackage{tabularx} % Wird benötig um Tabellen zu erstellen
\usepackage{graphicx} % Ermöglicht das einfügen von Bildern
\usepackage{geometry} % Dient zur Interface dimensionierung
\usepackage{pdflscape} % bessere Lesbarkeit des PDF Dokuments
\usepackage{pdfpages} % ermöglicht das importieren von PDF-Dateien
\usepackage{fancyhdr} % Ein schöner Header ist verfügbar
\usepackage{comment} % Mehrlinien-Kommentar werden möglich
\usepackage{listings} % Auflistung für Codeblocks
\usepackage{color} % Dient dem Syntaxhighlighting für Codeblocks
%\usepackage[usenames,dvipsnames]{xcolor} % spezial Farben
%\usepackage{menukeys}


% FARBEN
\definecolor{mygreen}{rgb}{0,0.6,0}
\definecolor{mygray}{rgb}{0.5,0.5,0.5}
\definecolor{mymauve}{rgb}{0.58,0,0.82}


% CAPTIONS
\usepackage{caption} % Ermöglicht einen schönen Header beim Code
\DeclareCaptionFont{white}{\color{white}}
\DeclareCaptionFormat{listing}{\colorbox{gray}{\parbox{\textwidth}{#1#2#3}}}
\captionsetup[lstlisting]{format=listing,labelfont=white,textfont=white}


% HYPERLINKS
\hypersetup{%
    colorlinks=false,% hyperlinks will be black
    pdfborderstyle={/S/U/W 0}% border style will be underline of width 1pt
}


% CODE LISTING
%\lstloadlanguages{Bash} %
\lstset{ %
    backgroundcolor=\color{white},
    basicstyle=\footnotesize,
    %basicstyle=\ttfamily\color{white} % Farbe des Code %
    breakatwhitespace=false,
    breaklines=true,
    captionpos=t,
    commentstyle=\color{mygreen},
    deletekeywords={...},
    escapeinside={\%*}{*)},
    extendedchars=true,
    %frame=single, %
    keepspaces=true,
    keywordstyle=\color{blue},
    % language=Bash, %
    morekeywords={*,...},
    % numbers=left, %
    % numbersep=5pt, %
    % numberstyle=\tiny\color{mygray}, %
    rulecolor=\color{mygray},
    showspaces=false,
    showstringspaces=false,
    showtabs=false,
    stepnumber=1,
    % stringstyle=\color{red}, %
    stringstyle=\color{mymauve},
    tabsize=2,
    title=\lstname
}
\lstset{literate=%
    {Ö}{{\"O}}1
    {Ä}{{\"A}}1
    {Ü}{{\"U}}1
    {ß}{{\ss}}2
    {ü}{{\"u}}1
    {ä}{{\"a}}1
    {ö}{{\"o}}1
}

% TABELLEN
\newcolumntype{b}{X} % Definition of columntypes for nice tables
\newcolumntype{s}{>{\hsize=.25\hsize}X}
%\newcommand{\heading}[1]{\multicolumn{1}{c}{#1}}

% HEADER
\setlength{\headheight}{13.6pt} % Höhe des Headers
\pagestyle{fancyplain}
\fancyhead[L]{\ipaAuthor} % Linke Seite
\fancyhead[C]{\today} % Mitte
\fancyhead[R]{\ipaFirma} % Rechts
