% VARIABELN
\newcommand{\ipaAuthor}{Max Muster}
\newcommand{\ipaFachvorgesetzter}{Erika Mustermann}
\newcommand{\ipaFirma}{Company}

% FASTER COMPILES
%\includeonly{part1}

% PACKAGES
\usepackage[utf8]{inputenc} % Ermöglicht Umlaute usw...
\usepackage[ngerman]{babel} % Deutsch-Support
\usepackage[ngerman]{translator}
\usepackage{makeidx} % Benötigt man für das Inhaltsverzeichnis
\usepackage{hyperref} % Ermöglicht Hyperlinks
\usepackage{tabularx} % Wird benötig um Tabellen zu erstellen
\usepackage{graphicx} % Ermöglicht das einfügen von Bildern
\usepackage{geometry} % Dient zur Interface dimensionierung
\usepackage{pdflscape} % bessere Lesbarkeit des PDF Dokuments
\usepackage{pdfpages} % ermöglicht das importieren von PDF-Dateien
\usepackage{fancyhdr} % Ein schöner Header ist verfügbar
\usepackage{comment} % Mehrlinien-Kommentar werden möglich
\usepackage{listings} % Auflistung für Codeblocks
\usepackage{color} % Dient dem Syntaxhighlighting für Codeblocks

\usepackage[outputdir=build]{minted} % Syntax highlighting für Code

%\usepackage[usenames,dvipsnames]{xcolor} % spezial Farben
%\usepackage{menukeys}


% FARBEN
\definecolor{mygreen}{rgb}{0,0.6,0}
\definecolor{mygray}{rgb}{0.5,0.5,0.5}
\definecolor{mymauve}{rgb}{0.58,0,0.82}


% HYPERLINKS
\hypersetup{%
    colorlinks=false,% hyperlinks will be black
    pdfborderstyle={/S/U/W 0}% border style will be underline of width 1pt
}


% HEADER
\setlength{\headheight}{13.6pt} % Höhe des Headers
\pagestyle{fancyplain}
\fancyhead[L]{\ipaAuthor} % Linke Seite
\fancyhead[C]{\today} % Mitte
\fancyhead[R]{\ipaFirma} % Rechts

% FOOTER
\addtolength{\textheight}{2cm}
