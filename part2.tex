\chapter{Initialisierung}

\begin{comment}
Die Initialisierung schafft eine definierte Ausgangslage für das Projekt und stellt sicher, dass die
Projektziele mit PkOrg übereinstimmen. Die Projektgrundlagen und der Projektauftrag sind
erarbeitet. Es wird ein Variantenentscheid getroffen, welcher schlussendlich von den Lernenden
realisiert und i.d.R eingeführt wird. Hinweis: Es ist sinnvoll, eine Risikoanalyse zu erstellen.
\end{comment}

%In diesem Abschnitt wird die Initialisierung behandelt und Abgeschlossen
\section{Studie Ist Zustand / Soll-Zustand}
\section{Vorgehensziele}
\begin{comment}
bsp.
Projekt beginn
Projekt ende
Zeitpensum
Meilensteine erreichen im Project
Projektmethode soll richtig angewendet werden
\end{comment}
\section{Systemziele}
\section{Anforderungen}
\section{Riskioanalyse}
\section{Varianten}
% Auswahlen der Varianten zum Projekt %
\section{Informationssicherheit und Datenschutz(ISDS)}

\begin{comment}
Die Initialisierung schafft eine definierte Ausgangslage für das Projekt und stellt sicher, dass die
Projektziele mit PkOrg übereinstimmen. Die Projektgrundlagen und der Projektauftrag sind
erarbeitet. Es wird ein Variantenentscheid getroffen, welcher schlussendlich von den Lernenden
realisiert und i.d.R eingeführt wird. Hinweis: Es ist sinnvoll, eine Risikoanalyse zu erstellen. 
\end{comment}
\chapter{Konzept}
\section{Detailkonzept}
\subsection{Lösungskonzept}
\begin{comment}
Hier muss schon relativ genau beschrieben werden wie das Netzwerk und die Hardware aussieht und wie die Lösung am Schluss abgeliefert werden soll.
\end{comment}
\subsection{Beschreibung der Variante}
\subsection{Anforderung an die Realisierung}
\subsection{Anforderungsabdeckung}
\section{Machbarkeitsanalyse}
\subsection{Wirtschaftliche Machbarkeit}
\subsection{Technische Machbarkeit}
\subsection{Ressourcen und Verfügbarkeit}
\subsection{Zeitliche Umsetzung}
\subsection{Empfehlung}
\section{Testkonzept}
\subsection{Testziel}
% Was soll dieser Test bringen und wie solle der grob Ablauf sein %
\subsection{Testobjekte}
% Objekt zb Apache %
\subsection{Testarten}
\subsection{Testabdeckung}
\subsection{Testrahmen}
\subsubsection{Testvorausetzungen}
\subsubsection{Fehlerklassen}
\subsubsection{Start- und Abbruchbedingungen}
\subsection{Testinfrastruktur}
\subsubsection{Testsystem}
\subsubsection{Testdaten}
\subsubsection{Testhilfsmittel}
\subsection{Testfallbeschreibung}
\subsection{Testplan}
\begin{comment}
Die in der Phase Initialisierung gewählte Variante wird konkretisiert sowie weitere Konzepte erstellt.
Die Ergebnisse werden so detailliert erarbeitet, dass eine aussenstehende Person (Experte) sämtliche
Schritte nachvollziehen kann. Es muss klar ersichtlich sein, was, wie, wo und wann realisiert wird.
\end{comment}
\chapter{Realisierung}

\begin{comment}
Das Produkt bzw. das IT-System wird realisiert und getestet. Die nötigen Vorarbeiten werden
geleistet, um die Einführungsrisiken zu minimieren. Braucht es noch ein „Re-Testing“ oder werden
mögliche kleine Fehler bei einem späteren Zeitpunkt noch korrigiert?
\end{comment}

\section{Realisierung Anwendung}
\subsection{Application}
\subsection{Application}
\subsection{Application}
\subsubsection{Application-subsection}
\subsubsection{Application-subsection}
\section{Tests durchführen}
\begin{comment}
- Die Testdurchführung erfolgt erst, wenn die Vorbedingungen dazu erfüllt sind. Entsprechende muss vorher die Testinfrastruktur freigegeben werden.
- Aktivitäten
- Prüfen, ob die Testvorbedingunge erfüllt sind, um die Tests zu starten
- Tests gemäss Testkonzept durchführen
- Testergebnisse protokollieren und gemäss Kriterien im Testkonzept beurteilen
- Gegebenfalls Fehler beheben und Tests wiederholen
- Vorgehen zu offenen Punkten vereinbaren
\end{comment}
\section{Benutzeranleitung für Systemadministratoren}

\begin{comment}
Das Produkt bzw. das IT-System wird realisiert und getestet. Die nötigen Vorarbeiten werden
geleistet, um die Einführungsrisiken zu minimieren. Braucht es noch ein „Re-Testing“ oder werden
mögliche kleine Fehler bei einem späteren Zeitpunkt noch korrigiert?
\end{comment}
\chapter{Einführung}

\begin{comment}
Der sichere Übergang vom alten zum neuen Zustand wird gewährleistet. Der Betrieb wird ggf.
aufgenommen und so lange durch das Projekt unterstützt, bis er stabil ist.
Die Dokumentationen werden pünktlich auf Pkorg hochgeladen.
Das Projekt wird abgeschlossen und die „Projektorganisation“ wird aufgelöst.
Danach folgen die Präsentation und die anschliessende Bewertung durch die Experten und
Fachvorgesetzten.
\end{comment}

% Hier können auch immer noch Tests durchgeführt werden
\section{Testkonzept und -infrastruktur überführen}
%Nach Projektabschluss werdne für Korrekturen und Weiterentwicklungen Tests druchgeführt. Deshalb werden das Testkonzept und die Testinfrastr
\chapter{Quellenverzeichnis}
\chapter{Glossar}

\begin{comment}
Der sichere Übergang vom alten zum neuen Zustand wird gewährleistet. Der Betrieb wird ggf.
aufgenommen und so lange durch das Projekt unterstützt, bis er stabil ist.
Die Dokumentationen werden pünktlich auf Pkorg hochgeladen.
Das Projekt wird abgeschlossen und die „Projektorganisation“ wird aufgelöst.
Danach folgen die Präsentation und die anschliessende Bewertung durch die Experten und
Fachvorgesetzten.
\end{comment}