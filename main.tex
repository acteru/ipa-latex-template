% IPA Dokumentation - Template
% Haupdatei - Main mit Paketen
\documentclass[a4paper,11pt,oneside]{report} % Dokumentklasse lege Grundeigenschaften des Dokuments fest
\usepackage[utf8]{inputenc} % Ermöglicht Umlaute usw...
\usepackage[ngerman]{babel} % Deutsch-Support
\usepackage{makeidx} % Benötigt man für das Inhaltsverzeichnis
\usepackage{hyperref} % Ermöglicht Hyperlinks
\usepackage{tabularx} % Wird benötig um Tabellen zu erstellen
\usepackage{graphicx} % Ermöglicht das einfügen von Bildern
\usepackage{geometry} % Dient zur Interface dimensionierung
\usepackage{pdflscape} % bessere lesbarkeit des Pdfdokuments
\usepackage{pdfpages} % ermöglicht das importieren von pdf-dateien
\usepackage{fancyhdr} % Ein schöner header ist verfügbar
%\usepackage[usenames,dvipsnames]{xcolor} % spezial Farben %
\usepackage{comment} % Mehrlinien-Kommentar werden möglich
\setlength{\headheight}{15.2pt}
\usepackage{caption} % Ermöglich einen schönen Capton-Header beim Code
\DeclareCaptionFont{white}{\color{white}}
\DeclareCaptionFormat{listing}{\colorbox{gray}{\parbox{\textwidth}{#1#2#3}}}
\captionsetup[lstlisting]{format=listing,labelfont=white,textfont=white}
\usepackage{listings} % Auflistung für Codeblocks
\usepackage{color} % Dient dem Syntaxhighlighting für Codeblocks 
%\usepackage{menukeys}
\hypersetup{%
  colorlinks=false,% hyperlinks will be black
  pdfborderstyle={/S/U/W 0}% border style will be underline of width 1pt
}
\definecolor{mygreen}{rgb}{0,0.6,0}
\definecolor{mygray}{rgb}{0.5,0.5,0.5}
\definecolor{mymauve}{rgb}{0.58,0,0.82}
%\lstloadlanguages{Bash} %
\lstset{ %
  backgroundcolor=\color{white},
  basicstyle=\footnotesize,
  %basicstyle=\ttfamily\color{white} % Farbe des Code %
  breakatwhitespace=false,
  breaklines=true,
  captionpos=t,
  commentstyle=\color{mygreen},
  deletekeywords={...},
  escapeinside={\%*}{*)},
  extendedchars=true,
   %frame=single, %
  keepspaces=true,
  keywordstyle=\color{blue},
 % language=Bash, %
  morekeywords={*,...},           
 % numbers=left, %
 % numbersep=5pt, %
 % numberstyle=\tiny\color{mygray}, %
  rulecolor=\color{mygray},
  showspaces=false,
  showstringspaces=false,
  showtabs=false,
  stepnumber=1,
 % stringstyle=\color{red}, %
  stringstyle=\color{mymauve},
  tabsize=2,
  title=\lstname
}
\lstset{literate=%
{Ö}{{\"O}}1
{Ä}{{\"A}}1
{Ü}{{\"U}}1
{ß}{{\ss}}2
{ü}{{\"u}}1
{ä}{{\"a}}1
{ö}{{\"o}}1
}
% Definition of columtypes for nice tables
\newcolumntype{b}{X}
\newcolumntype{s}{>{\hsize=.25\hsize}X}
%\newcommand{\heading}[1]{\multicolumn{1}{c}{#1}}
\setlength{\headheight}{13.6pt}
\pagestyle{fancyplain}
\fancyhead[L]{Max Muster}
\fancyhead[C]{\today}
\fancyhead[R]{Company}
\begin{document}
\begin{titlepage}
\begin{center}
\textsc{}\\[3cm]
\textsc{\LARGE placeholder}\\[1.5cm]
\textsc{\LARGE placeholder}\\[1.5cm]
\noindent
\begin{minipage}{0.4\textwidth}
\begin{flushleft} \large
\emph{Author:}\\
Max Muster
\end{flushleft}
\end{minipage}%
\begin{minipage}{0.4\textwidth}
\begin{flushright} \large
\emph{Fachvorgesetzter:} \\
Erika Mustermann
\end{flushright}
\end{minipage}
\vfill
% Bottom of the page
{\large \today}
\end{center}
\end{titlepage}
\section{Dokumentinformationen}
\nopagebreak
\section{Dokumentinformationen}

\subsection{Änderungskontrolle, Prüfung, Genehmigung}
\begin{tabular}{l | l | l | l}
    \textbf{Version} & \textbf{Datum} & \textbf{Name} & \textbf{Beschreibung}            \\
    1.0              & 26.01.2015     & Max Muster    & Dokumentenvorlage V1.0           \\
    1.1              & 27.01.2015     & Max Muster    & Dokumentation wird eröffnet V1.1 \\
\end{tabular}

\subsection{Referenzierte Dokumente}
\begin{itemize}
    \item placeholder item
\end{itemize}


\subsection{Verwendete Abkürzungen}
\begin{tabular}{l | l}
    \textbf{Abkürzung} & \textbf{Bedeutung} \\
    placeholder & placeholder \\
\end{tabular}

\pagebreak
\tableofcontents
\lstlistoflistings
\listoffigures
\listoftables
\part{Ablauf und Umfeld}
\chapter{Aufgabenstellung}
\section{Ausgangslage}
\section{Auftragsformulierung}
\section{Mittel und Methoden}
\section{Projektmanagementplan}
\section{Projektrollen}

\chapter{Vorkenntnisse}
% Auflistung der Kenntnisse [ Bereich - Gewichtung - Kommentar ]
\chapter{Vorarbeiten}
% Bereits geleistete vorarbeiten
\chapter{Firmenstandards}
\begin{lstlisting}[language=bash, caption=blabla]
#!/bin/bash
echo "Hallo world"
\end{lstlisting}
\section{Bootstrapping}
\section{KVM}
\section{IP-Adressen}
\section{Linux Standard Base}

\chapter{Organisation der IPA}
\section{Datensicherung der IPA}
\chapter{Zeitplan}

% ALTERNATIVE: Zeitplan per PDF importieren

\section{Erläuterung zum Zeitplan}
Der Zeitplan auf der nächsten Seite ist in eine Blockgrösse von zwei Stunden aufgeteilt. Blöcke von nur einer Stunde werden durch gleichzeitiges Arbeiten an zwei 2-Stundenblöcken dargestellt.

\newgeometry{bottom=2cm,top=2.5cm}

\newcommand{\soll}[3]{\ganttbar{#1}{#2}{#3}}
\newcommand{\ist}[2]{\ganttbar[bar top shift=-0.01,bar/.append style={bottom color=green}]{}{#1}{#2}}

\begin{landscape} % querformat für diese Seite
    \begin{ganttchart}[
            title height=1,
            y unit title=5mm,
            y unit chart=4mm,
            x unit=4.8mm,
            hgrid={dotted, draw=none},
            vgrid={draw=none, dotted, draw=none, black},
            % BAR
            bar label node/.append style={yshift=-1.3mm},
            bar height=.7,
            bar top shift=.3,
            bar/.append style={top color=lightgray!50, bottom color=gray},
            % MILESTONE
            milestone/.append style={top color=white, bottom color=gray},
            milestone height=0.4mm,
            milestone inline label node/.style={below=2mm, font=\scriptsize},
            milestone top shift=-0.15mm,
            %bar label node/.style={text width=4cm,align=right,anchor=east},
        ]{1}{40}
        \pgfcalendar{titlecal}{2016-05-06}{2016-05-24}{
            % Exclude weekends and mondays
            \ifdate{weekend}{}{\ifdate{Monday}{}{
                    \gantttitle{
                        \pgfcalendarweekdayshortname{\pgfcalendarcurrentweekday}
                        \pgfcalendarcurrentday.\pgfcalendarcurrentmonth
                    }{4}
                }
            }
        } \\
        \gantttitlelist{1,...,10}{4} \\
        \gantttitle{Initialisierung}{6}
        \gantttitle{Konzept}{12}
        \gantttitle{Realisierung}{16}
        \gantttitle{Abschluss}{6} \\

        %%%%%% EXPERTENBESUCHE
        \soll{Expertenbesuche}{7}{7}
        \soll{}{27}{27} \\
        \ist{7}{7}
        \ist{27}{27} \\
        %%%%%% JOURNAL & DOKUMENTATION
        \soll{Dokumentation / Journal}{2}{6}
        \soll{}{8}{18}
        \soll{}{23}{24}
        \soll{}{28}{28}
        \soll{}{32}{34}
        \soll{}{36}{36} \\
        \ist{2}{6}
        \ist{8}{18}
        \ist{20}{20}
        \ist{23}{24}
        \ist{28}{28}
        \ist{32}{34}
        \ist{36}{36}
        \ganttmilestone[inline]{M1}{6}
        \ganttmilestone[inline]{M2}{18} \\
        %%%%%% DAY ONE 1-4
        \soll{Zeitplan}{1}{1} \\
        \ist{1}{1} \\
        %%%%%% DAY TWO 5-8
        %%%%%% DAY THREE 9-12
        %%%%%% DAY FOUR 13-16
        %%%%%% DAY FIVE 17-20
        %%%%%% DAY SIX 21-24
        \soll{Scripts schreiben}{19}{22} \\
        \ist{19}{22} \\
        %%%%%% DAY SEVEN 25-28
        \soll{Product}{25}{26} \\
        \ist{25}{26} \\
        \soll{Product2}{27}{27} \\
        \ist{27}{27} \\
        %%%%%% DAY EIGHT 29-32
        \soll{Product3}{29}{29} \\
        \ist{29}{30} \\
        \soll{Testing}{30}{31} \\
        \ist{30}{31}
        \ganttmilestone[inline, milestone top shift=-0.4mm]{M3}{34} \\
        %%%%%% DAY NINE 33-36
        \soll{Reserve}{35}{35} \\
        \ist{35}{35} \\
        %%%%%% DAY TEN 37-40
        \soll{Abschluss}{37}{40} \\
        \ist{37}{40}
        \ganttmilestone[inline]{M4}{39}
    \end{ganttchart}

    {\addtolength{\leftskip}{5.1cm}
    \textbf{Legende}\\
    Grau: soll \\
    Grün: ist

    \textbf{Meilensteine}\\
    M1: Initialisierung abgeschlossen\\
    M2: Konzeptphase abgeschlossen\\
    M3: Realisierung abgeschlossen\\
    M4: Projekt abgeschlossen

}
\end{landscape}
\restoregeometry


\chapter{Arbeitsjournal}
\section{Tag 1: 27.01.2015 Dienstag }

\begin{table}[htb]
    \begin{tabularx}{\textwidth}{ Xcc }
        \hline
        \textbf{Tätigkeiten}                                                    & \textbf{Soll}                  & \textbf{Ist}                    \\ \hline
        \textbf{Tagesplan / Daily Scrum}                                        & 1h                             & 1h                              \\ \hline
        \textbf{Dokumentation Teil A:} Eröffnung der Dokumentation              & 2h                             & 2h                              \\ \hline
        \textbf{Dokumentation Teil A:} Erstellen des Zeitplans                  & 2h                             & 2h                              \\ \hline
        \textbf{Dokumentation Teil B und C:} Eröffnung der Projektdokumentation & 1h                             & 1h                              \\ \hline
        \textbf{Recherche,Vorbereitung:} placholder                             & 2h                             & 2h                              \\ \hline
    \end{tabularx}
\end{table}
\paragraph{Tagesablauf}
\paragraph{Hilfestellungen}
\paragraph{Reflexion}
\paragraph{Nächste Schritte}
\begin{itemize}
    \item placeholder 1
    \item placeholder 2
    \item placeholder 3
\end{itemize}
\newpage

\input{part1/journal/tag2}
\input{part1/journal/tag3}
\input{part1/journal/tag4}
\input{part1/journal/tag5}
\input{part1/journal/tag6}
\input{part1/journal/tag7}
\input{part1/journal/tag8}
\input{part1/journal/tag9}
\input{part1/journal/tag10}

\chapter{Abschlussbericht}
\section{Vergleich Ist/Soll}
% Verlief die Umsetzung wie geplant oder gab es Differenzen?

\section{Mittelbedarf}
% Welche Mittel wurden gebraucht oder mussten noch beschafft werden?

\section{Realisierungsbericht}
% Gab es Probleme während der Realisation? Kamen ungeplante Sachen zum Vorschein oder brauchte es Ergänzungen?

\section{Testbericht}
% Wie verliefen die Tests (erfolgreich / weniger erfolgreich)? Gab es schwerwiegende Fehler, oder solche, welche später korrigiert werden können/müssen?

\section{Fazit zum Projekt}
% Wie ist das Projekt verlaufen? (Objektive) Meinung. Hatte das Projekt Stolpersteine? Wenn ja: welche?

\section{Persönliches Fazit}
% Was war gut, was weniger? Was habe ich gelernt, was würde ich das nächste Mal anders machen? Ausführlich formulieren.

\chapter{Unterschriften Teil 1}
\begin{comment}
    Lernende und Fachvorgesetzte haben das Arbeitsjournal vor der Abgabe zu unterzeichnen und somit dessen Authentizität zu bezeugen. (Unterschrift auf dem gelben Deckblatt ist zwingend!!!).
    Der Form halber sollten jeweils Teil 1 und Teil 2 in jedem Dokument unterschreiben werden.
\end{comment}

\begin{tabularx}{0.9\textwidth}{| X | X | X |}
    \hline
    \textbf{Datum} & \textbf{Name/OE} & \textbf{Unterschrift} \\
    \hline
                   & & \\
                   & & \\
    \hline
                   & & \\
                   & & \\
    \hline
\end{tabularx}

\part{Projektdokumentation}
\chapter{Initialisierung}

\begin{comment}
Die Initialisierung schafft eine definierte Ausgangslage für das Projekt und stellt sicher, dass die
Projektziele mit PkOrg übereinstimmen. Die Projektgrundlagen und der Projektauftrag sind
erarbeitet. Es wird ein Variantenentscheid getroffen, welcher schlussendlich von den Lernenden
realisiert und i.d.R eingeführt wird. Hinweis: Es ist sinnvoll, eine Risikoanalyse zu erstellen.
\end{comment}

%In diesem Abschnitt wird die Initialisierung behandelt und Abgeschlossen
\section{Studie Ist Zustand / Soll-Zustand}
\section{Vorgehensziele}
\begin{comment}
bsp.
Projekt beginn
Projekt ende
Zeitpensum
Meilensteine erreichen im Project
Projektmethode soll richtig angewendet werden
\end{comment}
\section{Systemziele}
\section{Anforderungen}
\section{Riskioanalyse}
\section{Varianten}
% Auswahlen der Varianten zum Projekt %
\section{Informationssicherheit und Datenschutz(ISDS)}

\begin{comment}
Die Initialisierung schafft eine definierte Ausgangslage für das Projekt und stellt sicher, dass die
Projektziele mit PkOrg übereinstimmen. Die Projektgrundlagen und der Projektauftrag sind
erarbeitet. Es wird ein Variantenentscheid getroffen, welcher schlussendlich von den Lernenden
realisiert und i.d.R eingeführt wird. Hinweis: Es ist sinnvoll, eine Risikoanalyse zu erstellen. 
\end{comment}
\chapter{Konzept}
\section{Detailkonzept}
\subsection{Lösungskonzept}
\begin{comment}
Hier muss schon relativ genau beschrieben werden wie das Netzwerk und die Hardware aussieht und wie die Lösung am Schluss abgeliefert werden soll.
\end{comment}
\subsection{Beschreibung der Variante}
\subsection{Anforderung an die Realisierung}
\subsection{Anforderungsabdeckung}
\section{Machbarkeitsanalyse}
\subsection{Wirtschaftliche Machbarkeit}
\subsection{Technische Machbarkeit}
\subsection{Ressourcen und Verfügbarkeit}
\subsection{Zeitliche Umsetzung}
\subsection{Empfehlung}
\section{Testkonzept}
\subsection{Testziel}
% Was soll dieser Test bringen und wie solle der grob Ablauf sein %
\subsection{Testobjekte}
% Objekt zb Apache %
\subsection{Testarten}
\subsection{Testabdeckung}
\subsection{Testrahmen}
\subsubsection{Testvorausetzungen}
\subsubsection{Fehlerklassen}
\subsubsection{Start- und Abbruchbedingungen}
\subsection{Testinfrastruktur}
\subsubsection{Testsystem}
\subsubsection{Testdaten}
\subsubsection{Testhilfsmittel}
\subsection{Testfallbeschreibung}
\subsection{Testplan}
\begin{comment}
Die in der Phase Initialisierung gewählte Variante wird konkretisiert sowie weitere Konzepte erstellt.
Die Ergebnisse werden so detailliert erarbeitet, dass eine aussenstehende Person (Experte) sämtliche
Schritte nachvollziehen kann. Es muss klar ersichtlich sein, was, wie, wo und wann realisiert wird.
\end{comment}
\chapter{Realisierung}

\begin{comment}
Das Produkt bzw. das IT-System wird realisiert und getestet. Die nötigen Vorarbeiten werden
geleistet, um die Einführungsrisiken zu minimieren. Braucht es noch ein „Re-Testing“ oder werden
mögliche kleine Fehler bei einem späteren Zeitpunkt noch korrigiert?
\end{comment}

\section{Realisierung Anwendung}
\subsection{Application}
\subsection{Application}
\subsection{Application}
\subsubsection{Application-subsection}
\subsubsection{Application-subsection}
\section{Tests durchführen}
\begin{comment}
- Die Testdurchführung erfolgt erst, wenn die Vorbedingungen dazu erfüllt sind. Entsprechende muss vorher die Testinfrastruktur freigegeben werden.
- Aktivitäten
- Prüfen, ob die Testvorbedingunge erfüllt sind, um die Tests zu starten
- Tests gemäss Testkonzept durchführen
- Testergebnisse protokollieren und gemäss Kriterien im Testkonzept beurteilen
- Gegebenfalls Fehler beheben und Tests wiederholen
- Vorgehen zu offenen Punkten vereinbaren
\end{comment}
\section{Benutzeranleitung für Systemadministratoren}

\begin{comment}
Das Produkt bzw. das IT-System wird realisiert und getestet. Die nötigen Vorarbeiten werden
geleistet, um die Einführungsrisiken zu minimieren. Braucht es noch ein „Re-Testing“ oder werden
mögliche kleine Fehler bei einem späteren Zeitpunkt noch korrigiert?
\end{comment}
\chapter{Einführung}

\begin{comment}
Der sichere Übergang vom alten zum neuen Zustand wird gewährleistet. Der Betrieb wird ggf.
aufgenommen und so lange durch das Projekt unterstützt, bis er stabil ist.
Die Dokumentationen werden pünktlich auf Pkorg hochgeladen.
Das Projekt wird abgeschlossen und die „Projektorganisation“ wird aufgelöst.
Danach folgen die Präsentation und die anschliessende Bewertung durch die Experten und
Fachvorgesetzten.
\end{comment}

% Hier können auch immer noch Tests durchgeführt werden
\section{Testkonzept und -infrastruktur überführen}
%Nach Projektabschluss werdne für Korrekturen und Weiterentwicklungen Tests druchgeführt. Deshalb werden das Testkonzept und die Testinfrastr
\chapter{Quellenverzeichnis}
\chapter{Glossar}

\begin{comment}
Der sichere Übergang vom alten zum neuen Zustand wird gewährleistet. Der Betrieb wird ggf.
aufgenommen und so lange durch das Projekt unterstützt, bis er stabil ist.
Die Dokumentationen werden pünktlich auf Pkorg hochgeladen.
Das Projekt wird abgeschlossen und die „Projektorganisation“ wird aufgelöst.
Danach folgen die Präsentation und die anschliessende Bewertung durch die Experten und
Fachvorgesetzten.
\end{comment}
\part{Anhang}
\chapter{Step-By-Step Anleitung}
\end{document}